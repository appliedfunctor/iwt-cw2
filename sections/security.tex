\section{Security concerns}

UPnP provides an ease of use in auto-configuring network devices. The cost of this convenient ease of use is security. Despite the UPnP forum publishing a design document for implementing device security in 2003\cite{upnp-security}, devices are not required to implement any security in order to be UPnP compliant\cite{selen2006upnp} \cite{haque2007upnp} and most do not implement the security design document\cite{selen2006upnp}.

A number of security concerns have arisen, some due to the design of UPnP itself and others due to its implementations, such as one from 2002 where the Windows XP UPnP implementation was vulnerable to a remotely executable buffer overflow and both a Distributed (DDOS) and non-Distributed (DOS) Denial of Service attack\cite{winxp-upnp-flaw} \cite{haque2007upnp}.

Another widespread security concern of UPnP was a result of routers which suffered from a flaw whereby they acted on UPnP control requests from the Wide Area Network (WAN)\cite{wan} as well as the Local Area Network (LAN)\cite{lan}. This allowed an attacker to open ports on the router from outside the network using UPnP control requests.

While this was not an issue with UPnP itself, the architecture of UPnP provides security concerns. By design, all nodes on the LAN are trusted, so any compromised device is capable of compromising the rest of the network. Even when excluding control points, which are the only devices covered by the UPnP forum security document\cite{upnp-security}, all devices will respond to search and information requests made under discovery which can help to identify devices through their manufacturer and device information.

The effect of absent security in UPnP devices, particularly the router or gateway devices is the ability to alter DNS to re-route traffic to a different site, perform man-in-the-middle attacks\cite{man-in-the-middle} and hijack devices for DOS and DDOS attacks\cite{akamai-ddos}.